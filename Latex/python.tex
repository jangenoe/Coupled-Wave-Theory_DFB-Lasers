%% Generated by Sphinx.
\def\sphinxdocclass{jupyterBook}
\documentclass[a4paper,10pt,english,openany,oneside]{jupyterBook}
\ifdefined\pdfpxdimen
   \let\sphinxpxdimen\pdfpxdimen\else\newdimen\sphinxpxdimen
\fi \sphinxpxdimen=.75bp\relax
\ifdefined\pdfimageresolution
    \pdfimageresolution= \numexpr \dimexpr1in\relax/\sphinxpxdimen\relax
\fi
%% let collapsible pdf bookmarks panel have high depth per default
\PassOptionsToPackage{bookmarksdepth=5}{hyperref}
%% turn off hyperref patch of \index as sphinx.xdy xindy module takes care of
%% suitable \hyperpage mark-up, working around hyperref-xindy incompatibility
\PassOptionsToPackage{hyperindex=false}{hyperref}
%% memoir class requires extra handling
\makeatletter\@ifclassloaded{memoir}
{\ifdefined\memhyperindexfalse\memhyperindexfalse\fi}{}\makeatother

\PassOptionsToPackage{warn}{textcomp}

\catcode`^^^^00a0\active\protected\def^^^^00a0{\leavevmode\nobreak\ }
\usepackage{cmap}
\usepackage{fontspec}
\defaultfontfeatures[\rmfamily,\sffamily,\ttfamily]{}
\usepackage{amsmath,amssymb,amstext}
\usepackage{polyglossia}
\setmainlanguage{english}



\setmainfont{FreeSerif}[
  Extension      = .otf,
  UprightFont    = *,
  ItalicFont     = *Italic,
  BoldFont       = *Bold,
  BoldItalicFont = *BoldItalic
]
\setsansfont{FreeSans}[
  Extension      = .otf,
  UprightFont    = *,
  ItalicFont     = *Oblique,
  BoldFont       = *Bold,
  BoldItalicFont = *BoldOblique,
]
\setmonofont{FreeMono}[
  Extension      = .otf,
  UprightFont    = *,
  ItalicFont     = *Oblique,
  BoldFont       = *Bold,
  BoldItalicFont = *BoldOblique,
]



\usepackage[Bjarne]{fncychap}
\usepackage[,numfigreset=1,mathnumfig]{sphinx}

\fvset{fontsize=\small}
\usepackage{geometry}


% Include hyperref last.
\usepackage{hyperref}
% Fix anchor placement for figures with captions.
\usepackage{hypcap}% it must be loaded after hyperref.
% Set up styles of URL: it should be placed after hyperref.
\urlstyle{same}


\usepackage{sphinxmessages}


\usepackage{etoolbox}
\AtBeginEnvironment{figure}{\pretocmd{\hyperlink}{\protect}{}{}}  

        % Start of preamble defined in sphinx-jupyterbook-latex %
         \usepackage[Latin,Greek]{ucharclasses}
        \usepackage{unicode-math}
        % fixing title of the toc
        \addto\captionsenglish{\renewcommand{\contentsname}{Contents}}
        \hypersetup{
            pdfencoding=auto,
            psdextra
        }
        % End of preamble defined in sphinx-jupyterbook-latex %
        

\title{Coupled Wave Theory of Distributed Feedback Lasers}
\date{Mar 23, 2023}
\release{}
\author{Jan Genoe, Iakov Goldberg, Nirav Annavarapu, Simon Leitner, Robert Gehlhaar}
\newcommand{\sphinxlogo}{\sphinxincludegraphics{logo.png}\par}
\renewcommand{\releasename}{}
\makeindex
\begin{document}

\pagestyle{empty}
\sphinxmaketitle
\pagestyle{plain}
\sphinxtableofcontents
\pagestyle{normal}
\phantomsection\label{\detokenize{intro::doc}}


\sphinxAtStartPar
This jupyter book applies the calculations of Kogelnik and Shank in J. Appl. Phys. 43, 2327 (1972) {[}\hyperlink{cite.bib:id3}{1}{]} to a few use cases.
The calculations have been validated by replotting the graphs from {[}\hyperlink{cite.bib:id3}{1}{]}.
Each of the chapters elaborates one use case.

\sphinxAtStartPar
The online version allows to recalculate all graphs in \sphinxhref{https://jangenoe.github.io/Coupled-Wave-Theory\_DFB-Lasers/Lite}{jupyter\sphinxhyphen{}lite} (even without any code installed).

\sphinxAtStartPar
All code is available on \sphinxhref{https://github.com/jangenoe/Coupled-Wave-Theory\_DFB-Lasers}{Github}

\sphinxstepscope


\chapter{Coupled‐Wave Theory of Distributed Feedback Lasers}
\label{\detokenize{Kogelnik-Shank_Coupled-Wave-Theory_DFB-Lasers:coupledwave-theory-of-distributed-feedback-lasers}}\label{\detokenize{Kogelnik-Shank_Coupled-Wave-Theory_DFB-Lasers::doc}}
\sphinxAtStartPar
This page comprises simulations from the manuscript H. Kogelnik and C. V. Shank, \sphinxstyleemphasis{Coupled‐Wave Theory of Distributed Feedback Lasers}. Journal of Applied Physics, \sphinxstylestrong{43} (1972) 2327–2335. \sphinxhref{https://doi.org/10.1063/1.1661499}{doi:10.1063/1.1661499}

\sphinxAtStartPar
We compare the different simulations in the paper with our code.

\begin{sphinxuseclass}{cell}
\end{sphinxuseclass}
\begin{sphinxuseclass}{cell}
\end{sphinxuseclass}

\section{Dispersion diagram for index modulation for various gain to coupling parameter ratios}
\label{\detokenize{Kogelnik-Shank_Coupled-Wave-Theory_DFB-Lasers:dispersion-diagram-for-index-modulation-for-various-gain-to-coupling-parameter-ratios}}
\sphinxAtStartPar
\hyperref[\detokenize{Kogelnik-Shank_Coupled-Wave-Theory_DFB-Lasers:kogelnik1}]{Fig.\@ \ref{\detokenize{Kogelnik-Shank_Coupled-Wave-Theory_DFB-Lasers:kogelnik1}}} calculates the dispersion diagram for index modulation for various gain (\(\alpha_o\)) to coupling (\(\kappa\)) parameter ratios. In case of index modulation we have that \(\kappa= \pi n_1/\lambda_o\). We observe that the calculated result fits the result from {[}\hyperlink{cite.bib:id3}{1}{]}, as can be seen in \hyperref[\detokenize{Kogelnik-Shank_Coupled-Wave-Theory_DFB-Lasers:kogelnik2}]{Fig.\@ \ref{\detokenize{Kogelnik-Shank_Coupled-Wave-Theory_DFB-Lasers:kogelnik2}}}.

\begin{sphinxuseclass}{cell}\begin{sphinxVerbatimOutput}

\begin{sphinxuseclass}{cell_output}
\begin{figure}[htbp]
\centering
\capstart

\noindent\sphinxincludegraphics{{19a6cc7e1fd073f0d4c165d56e37007f1f055eb8bce23f54943d10f06c612e70}.png}
\caption{Calculated dispersion diagram for index modulation for various gain to coupling parameter ratios, right: corresponding graph from from {[}\hyperlink{cite.bib:id3}{1}{]}}\label{\detokenize{Kogelnik-Shank_Coupled-Wave-Theory_DFB-Lasers:kogelnik1}}\end{figure}

\end{sphinxuseclass}\end{sphinxVerbatimOutput}

\end{sphinxuseclass}
\begin{figure}[htbp]
\centering
\capstart

\noindent\sphinxincludegraphics[width=1.000\linewidth]{{Kogelnik2}.png}
\caption{Dispersion diagram for index modulation for various gain to coupling parameter ratios}\label{\detokenize{Kogelnik-Shank_Coupled-Wave-Theory_DFB-Lasers:kogelnik2}}\end{figure}


\section{Dispersion diagram for gain modulation for various gain to coupling parameter ratios}
\label{\detokenize{Kogelnik-Shank_Coupled-Wave-Theory_DFB-Lasers:dispersion-diagram-for-gain-modulation-for-various-gain-to-coupling-parameter-ratios}}
\sphinxAtStartPar
\hyperref[\detokenize{Kogelnik-Shank_Coupled-Wave-Theory_DFB-Lasers:kogelnik3}]{Fig.\@ \ref{\detokenize{Kogelnik-Shank_Coupled-Wave-Theory_DFB-Lasers:kogelnik3}}} calculates the dispersion diagram for index modulation for various gain (\(\alpha_o\)) to coupling (\(\kappa\)) parameter ratios. In case of index modulation we have that \(\kappa= \frac{1}{2} j \alpha_1\). We observe that the calculated result fits \hyperref[\detokenize{Kogelnik-Shank_Coupled-Wave-Theory_DFB-Lasers:kogelnik4}]{Fig.\@ \ref{\detokenize{Kogelnik-Shank_Coupled-Wave-Theory_DFB-Lasers:kogelnik4}}}.

\begin{sphinxuseclass}{cell}\begin{sphinxVerbatimOutput}

\begin{sphinxuseclass}{cell_output}
\begin{figure}[htbp]
\centering
\capstart

\noindent\sphinxincludegraphics{{3472c4a6070faaa29e840520a08d1f67c754d908a7239cbc3d978cb2ffbfcb5e}.png}
\caption{Calculated dispersion diagram for gain modulation for various gain to coupling parameter ratios}\label{\detokenize{Kogelnik-Shank_Coupled-Wave-Theory_DFB-Lasers:kogelnik3}}\end{figure}

\end{sphinxuseclass}\end{sphinxVerbatimOutput}

\end{sphinxuseclass}
\begin{figure}[htbp]
\centering
\capstart

\noindent\sphinxincludegraphics[width=1.000\linewidth]{{Kogelnik4}.png}
\caption{Dispersion diagram for gain modulation for various gain to coupling parameter ratios}\label{\detokenize{Kogelnik-Shank_Coupled-Wave-Theory_DFB-Lasers:kogelnik4}}\end{figure}


\section{Mode spectrum for index coupling}
\label{\detokenize{Kogelnik-Shank_Coupled-Wave-Theory_DFB-Lasers:mode-spectrum-for-index-coupling}}
\begin{sphinxuseclass}{cell}\begin{sphinxVerbatimOutput}

\begin{sphinxuseclass}{cell_output}
\begin{figure}[htbp]
\centering
\capstart

\noindent\sphinxincludegraphics{{b6274490a4592069f8ecfe78645865c62686a6022af4fc79873a2acab088a9d9}.png}
\caption{Calculated gain required for threshold vs frequency deviation from the Bragg condition for index modulation}\label{\detokenize{Kogelnik-Shank_Coupled-Wave-Theory_DFB-Lasers:kogelnik5c}}\end{figure}

\end{sphinxuseclass}\end{sphinxVerbatimOutput}

\end{sphinxuseclass}
\begin{figure}[htbp]
\centering
\capstart

\noindent\sphinxincludegraphics[width=1.000\linewidth]{{Kogelnik5}.png}
\caption{Plot of the gain required for threshold vs frequency deviation from the Bragg condition for an index modulation. Only half of the spectrum is shown because of symmetry.}\label{\detokenize{Kogelnik-Shank_Coupled-Wave-Theory_DFB-Lasers:kogelnik5}}\end{figure}


\section{Mode spectrum for gain coupling}
\label{\detokenize{Kogelnik-Shank_Coupled-Wave-Theory_DFB-Lasers:mode-spectrum-for-gain-coupling}}
\begin{sphinxuseclass}{cell}\begin{sphinxVerbatimOutput}

\begin{sphinxuseclass}{cell_output}
\begin{figure}[htbp]
\centering
\capstart

\noindent\sphinxincludegraphics{{994232cdc4e71ef98a9da10e7d86c1b0189a004f70f02dd1c29e1ff11e82f5c0}.png}
\caption{Calculated DC gain required for threshold vs frequency deviation from the Bragg condition for gain modulation}\label{\detokenize{Kogelnik-Shank_Coupled-Wave-Theory_DFB-Lasers:kogelnik7c}}\end{figure}

\end{sphinxuseclass}\end{sphinxVerbatimOutput}

\end{sphinxuseclass}
\begin{figure}[htbp]
\centering
\capstart

\noindent\sphinxincludegraphics[width=1.000\linewidth]{{Kogelnik7}.png}
\caption{Plot of the DC gain required for threshold vs frequency deviation from the Bragg condition for gain modulation. Only half of the spectrum is shown because of symmetry.}\label{\detokenize{Kogelnik-Shank_Coupled-Wave-Theory_DFB-Lasers:kogelnik7}}\end{figure}

\begin{sphinxuseclass}{cell}\begin{sphinxVerbatimOutput}

\begin{sphinxuseclass}{cell_output}
\begin{figure}[htbp]
\centering
\capstart

\noindent\sphinxincludegraphics{{60b727f3b722ec25ebb518911a7275af8c4a377449d533211e3808eab2f9edcb}.png}
\caption{Calculated gain at threshold vs coupling strength for various modes}\label{\detokenize{Kogelnik-Shank_Coupled-Wave-Theory_DFB-Lasers:kogelnik8c}}\end{figure}

\end{sphinxuseclass}\end{sphinxVerbatimOutput}

\end{sphinxuseclass}
\begin{figure}[htbp]
\centering
\capstart

\noindent\sphinxincludegraphics[width=1.000\linewidth]{{Kogelnik8}.png}
\caption{Plot of the gain at threshold vs coupling strength for various modes. The mode number N refers to a set of modes placed symmetrically about the Bragg frequency.}\label{\detokenize{Kogelnik-Shank_Coupled-Wave-Theory_DFB-Lasers:kogelnik8}}\end{figure}

\begin{sphinxuseclass}{cell}\begin{sphinxVerbatimOutput}

\begin{sphinxuseclass}{cell_output}
\begin{figure}[htbp]
\centering
\capstart

\noindent\sphinxincludegraphics{{da546e0693eb8c999df21cc837f1bd002aea638d1bb918dd872d646a2c789ac6}.png}
\caption{Calculated gain at threshold vs coupling strength for various modes}\label{\detokenize{Kogelnik-Shank_Coupled-Wave-Theory_DFB-Lasers:kogelnik9c}}\end{figure}

\end{sphinxuseclass}\end{sphinxVerbatimOutput}

\end{sphinxuseclass}
\begin{figure}[htbp]
\centering
\capstart

\noindent\sphinxincludegraphics[width=1.000\linewidth]{{Kogelnik9}.png}
\caption{Plot of the gain at threshold vs coupling strength for various modes. The N=O mode corresponds to a mode at the Bragg frequency. The numbers N>O correspond to a set of modes symmetrically placed about the Bragg frequency.}\label{\detokenize{Kogelnik-Shank_Coupled-Wave-Theory_DFB-Lasers:kogelnik9}}\end{figure}

\begin{sphinxuseclass}{cell}\begin{sphinxVerbatimOutput}

\begin{sphinxuseclass}{cell_output}
\begin{figure}[htbp]
\centering
\capstart

\noindent\sphinxincludegraphics{{50bf4acd7e4bd72b498c6ac0792b865da712f1e129ff39bc53d59d76c3ff6762}.png}
\caption{Calculated gain at threshold vs coupling strength for various modes}\label{\detokenize{Kogelnik-Shank_Coupled-Wave-Theory_DFB-Lasers:kogelnik10c}}\end{figure}

\end{sphinxuseclass}\end{sphinxVerbatimOutput}

\end{sphinxuseclass}
\begin{figure}[htbp]
\centering
\capstart

\noindent\sphinxincludegraphics[width=1.000\linewidth]{{Kogelnik10}.png}
\caption{Plot of the spatial intensity distribution of the lowest order modes at various coupling levels.}\label{\detokenize{Kogelnik-Shank_Coupled-Wave-Theory_DFB-Lasers:kogelnik10}}\end{figure}

\begin{figure}[htbp]
\centering
\capstart

\noindent\sphinxincludegraphics{{Kogelnik11a}.png}
\caption{Plot of the spatial intensity distribution for the first three modes at \(\pi n_i L/\lambda=0.25\).}\label{\detokenize{Kogelnik-Shank_Coupled-Wave-Theory_DFB-Lasers:kogelnik11a}}\end{figure}

\begin{sphinxuseclass}{cell}\begin{sphinxVerbatimOutput}

\begin{sphinxuseclass}{cell_output}
\begin{figure}[htbp]
\centering
\capstart

\noindent\sphinxincludegraphics{{e145bacd340bcef303b6998539b1ddc7b6bb350fb194453c1db94d39d6f9b677}.png}
\caption{Calculated spatial intensity distribution for the first three modes}\label{\detokenize{Kogelnik-Shank_Coupled-Wave-Theory_DFB-Lasers:kogelnik11ac}}\end{figure}

\end{sphinxuseclass}\end{sphinxVerbatimOutput}

\end{sphinxuseclass}
\begin{sphinxuseclass}{cell}\begin{sphinxVerbatimOutput}

\begin{sphinxuseclass}{cell_output}
\begin{figure}[htbp]
\centering
\capstart

\noindent\sphinxincludegraphics{{74cb0d262effcaa975755df1c5cce830d0ba5d8006edc72d9b55a73c17604bd4}.png}
\caption{Calculated spatial intensity distribution for the first three modes}\label{\detokenize{Kogelnik-Shank_Coupled-Wave-Theory_DFB-Lasers:kogelnik11bc}}\end{figure}

\end{sphinxuseclass}\end{sphinxVerbatimOutput}

\end{sphinxuseclass}
\begin{figure}[htbp]
\centering
\capstart

\noindent\sphinxincludegraphics{{Kogelnik11b}.png}
\caption{Plot of the spatial intensity distribution for the first three modes at \(\pi n_i L/\lambda=1\).}\label{\detokenize{Kogelnik-Shank_Coupled-Wave-Theory_DFB-Lasers:kogelnik11b}}\end{figure}

\begin{sphinxuseclass}{cell}\begin{sphinxVerbatimOutput}

\begin{sphinxuseclass}{cell_output}
\begin{figure}[htbp]
\centering
\capstart

\noindent\sphinxincludegraphics{{6069f76d2b45a0ada451b8994c8e075f6eb2f58fc2c382c986c586dab4080840}.png}
\caption{Calculated spatial intensity distribution for the first three modes}\label{\detokenize{Kogelnik-Shank_Coupled-Wave-Theory_DFB-Lasers:kogelnik11cc}}\end{figure}

\end{sphinxuseclass}\end{sphinxVerbatimOutput}

\end{sphinxuseclass}
\begin{figure}[htbp]
\centering
\capstart

\noindent\sphinxincludegraphics{{Kogelnik11c}.png}
\caption{Plot of the spatial intensity distribution for the first three modes at \(\pi n_i L/\lambda=2\).}\label{\detokenize{Kogelnik-Shank_Coupled-Wave-Theory_DFB-Lasers:kogelnik11c}}\end{figure}

\begin{sphinxuseclass}{cell}\begin{sphinxVerbatimOutput}

\begin{sphinxuseclass}{cell_output}
\begin{figure}[htbp]
\centering
\capstart

\noindent\sphinxincludegraphics{{46c85b98847252236e4ab7db10f1b447e4fa2947a2b9f15045e4d88a7bc18136}.png}
\caption{Calculated spatial intensity distribution for the first three modes}\label{\detokenize{Kogelnik-Shank_Coupled-Wave-Theory_DFB-Lasers:kogelnik11dc}}\end{figure}

\end{sphinxuseclass}\end{sphinxVerbatimOutput}

\end{sphinxuseclass}
\begin{figure}[htbp]
\centering
\capstart

\noindent\sphinxincludegraphics{{Kogelnik11d}.png}
\caption{Plot of the spatial intensity distribution for the first three modes at \(\pi n_i L/\lambda=10\).}\label{\detokenize{Kogelnik-Shank_Coupled-Wave-Theory_DFB-Lasers:kogelnik11d}}\end{figure}

\begin{figure}[htbp]
\centering
\capstart

\noindent\sphinxincludegraphics{{Kogelnik12a}.png}
\caption{Plot of the spatial intensity distribution for the first three modes at \(\alpha L / 2 =0.25\).}\label{\detokenize{Kogelnik-Shank_Coupled-Wave-Theory_DFB-Lasers:kogelnik12a}}\end{figure}

\begin{sphinxuseclass}{cell}\begin{sphinxVerbatimOutput}

\begin{sphinxuseclass}{cell_output}
\begin{figure}[htbp]
\centering
\capstart

\noindent\sphinxincludegraphics{{9af254fc974b2960d0a6ae9c5cde1d8ce1c2e3c43095fdb7013a63af778cde2d}.png}
\caption{Calculated spatial intensity distribution for the first three modes at  L / 2 =1}\label{\detokenize{Kogelnik-Shank_Coupled-Wave-Theory_DFB-Lasers:kogelnik12bc}}\end{figure}

\end{sphinxuseclass}\end{sphinxVerbatimOutput}

\end{sphinxuseclass}
\begin{figure}[htbp]
\centering
\capstart

\noindent\sphinxincludegraphics{{Kogelnik12b}.png}
\caption{Plot of the spatial intensity distribution for the first three modes at \(\alpha L / 2 =1\).}\label{\detokenize{Kogelnik-Shank_Coupled-Wave-Theory_DFB-Lasers:kogelnik12b}}\end{figure}

\begin{sphinxuseclass}{cell}\begin{sphinxVerbatimOutput}

\begin{sphinxuseclass}{cell_output}
\begin{figure}[htbp]
\centering
\capstart

\noindent\sphinxincludegraphics{{55a8aac2b728baee58cc8290cb39de2d5e57e5d0bf252a901d16bd025b4a5e2f}.png}
\caption{Calculated spatial intensity distribution for the first three modes at  L / 2 =2}\label{\detokenize{Kogelnik-Shank_Coupled-Wave-Theory_DFB-Lasers:kogelnik12cc}}\end{figure}

\end{sphinxuseclass}\end{sphinxVerbatimOutput}

\end{sphinxuseclass}
\begin{figure}[htbp]
\centering
\capstart

\noindent\sphinxincludegraphics{{Kogelnik12c}.png}
\caption{Plot of the spatial intensity distribution for the first three modes at \(\alpha L / 2 =2\).}\label{\detokenize{Kogelnik-Shank_Coupled-Wave-Theory_DFB-Lasers:kogelnik12c}}\end{figure}

\begin{sphinxuseclass}{cell}\begin{sphinxVerbatimOutput}

\begin{sphinxuseclass}{cell_output}
\begin{figure}[htbp]
\centering
\capstart

\noindent\sphinxincludegraphics{{41cdf44d5ebe76b878d43fa4cccdc97fdf293ece56b43a5f330b8cafa33e85ba}.png}
\caption{Calculated spatial intensity distribution for the first three modes at  L / 2 =10}\label{\detokenize{Kogelnik-Shank_Coupled-Wave-Theory_DFB-Lasers:kogelnik12dc}}\end{figure}

\end{sphinxuseclass}\end{sphinxVerbatimOutput}

\end{sphinxuseclass}
\begin{figure}[htbp]
\centering
\capstart

\noindent\sphinxincludegraphics{{Kogelnik12d}.png}
\caption{Plot of the spatial intensity distribution for the first three modes at \(\alpha L / 2 =10\).}\label{\detokenize{Kogelnik-Shank_Coupled-Wave-Theory_DFB-Lasers:kogelnik12d}}\end{figure}

\sphinxstepscope


\chapter{UV\sphinxhyphen{}Nanoimprinted Distributed\sphinxhyphen{}Feedback Perovskite Lasing}
\label{\detokenize{NanoimprintedDFB:uv-nanoimprinted-distributed-feedback-perovskite-lasing}}\label{\detokenize{NanoimprintedDFB::doc}}
\sphinxAtStartPar
This chapter comprises simulations from the manuscript:

\sphinxAtStartPar
Iakov Goldberg, Nirav Annavarapu, Simon Leitner, Karim Elkhouly, Fei Han, Tibor Kuna, Weiming Qiu, Cedric Rolin, Jan Genoe, Robert Gehlhaar and Paul Heremans, \sphinxstyleemphasis{“Multimode Lasing in All\sphinxhyphen{}Solution\sphinxhyphen{}Processed UV\sphinxhyphen{}Nanoimprinted Distributed Feedback MAPbI$_{\text{3}}$ Perovskite Waveguides”}, submitted manuscript

\sphinxAtStartPar
The calculation of the modes is done in accordance to {[}\hyperlink{cite.bib:id3}{1}{]}.

\begin{sphinxuseclass}{cell}
\end{sphinxuseclass}
\begin{sphinxuseclass}{cell}\begin{sphinxVerbatimOutput}

\begin{sphinxuseclass}{cell_output}
\begin{figure}[htbp]
\centering
\capstart

\noindent\sphinxincludegraphics{{1ebcf8af2a8ef5af869371d1fe3ed947532972361e69aa275d2eeef3280d24c8}.png}
\caption{Upper left: calculated dispersion relation prior to lasing, Upper right: intensity of the lasing mode in the near field, lower left: required DC gain for the lasing threshold, the dotted lines show the measured lasing lines at 3 different locations, lower right: far field lasing intensity as a function of the angle}\label{\detokenize{NanoimprintedDFB:iakov1}}\end{figure}

\end{sphinxuseclass}\end{sphinxVerbatimOutput}

\end{sphinxuseclass}
\sphinxstepscope


\chapter{References}
\label{\detokenize{bib:references}}\label{\detokenize{bib::doc}}
\begin{sphinxthebibliography}{1}
\bibitem[1]{bib:id3}
\sphinxAtStartPar
H. Kogelnik and C. V. Shank. Coupled\sphinxhyphen{}Wave Theory of Distributed Feedback Lasers. \sphinxstyleemphasis{Journal of Applied Physics}, 43(5):2327–2335, May 1972. \sphinxhref{https://doi.org/10.1063/1.1661499}{doi:10.1063/1.1661499}.
\end{sphinxthebibliography}







\renewcommand{\indexname}{Index}
\printindex
\end{document}