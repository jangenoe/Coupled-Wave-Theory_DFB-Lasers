%% Generated by Sphinx.
\def\sphinxdocclass{jupyterBook}
\documentclass[a4paper,10pt,english,openany,oneside]{jupyterBook}
\ifdefined\pdfpxdimen
   \let\sphinxpxdimen\pdfpxdimen\else\newdimen\sphinxpxdimen
\fi \sphinxpxdimen=.75bp\relax
\ifdefined\pdfimageresolution
    \pdfimageresolution= \numexpr \dimexpr1in\relax/\sphinxpxdimen\relax
\fi
%% let collapsible pdf bookmarks panel have high depth per default
\PassOptionsToPackage{bookmarksdepth=5}{hyperref}
%% turn off hyperref patch of \index as sphinx.xdy xindy module takes care of
%% suitable \hyperpage mark-up, working around hyperref-xindy incompatibility
\PassOptionsToPackage{hyperindex=false}{hyperref}
%% memoir class requires extra handling
\makeatletter\@ifclassloaded{memoir}
{\ifdefined\memhyperindexfalse\memhyperindexfalse\fi}{}\makeatother

\PassOptionsToPackage{warn}{textcomp}

\catcode`^^^^00a0\active\protected\def^^^^00a0{\leavevmode\nobreak\ }
\usepackage{cmap}
\usepackage{fontspec}
\defaultfontfeatures[\rmfamily,\sffamily,\ttfamily]{}
\usepackage{amsmath,amssymb,amstext}
\usepackage{polyglossia}
\setmainlanguage{english}



\setmainfont{FreeSerif}[
  Extension      = .otf,
  UprightFont    = *,
  ItalicFont     = *Italic,
  BoldFont       = *Bold,
  BoldItalicFont = *BoldItalic
]
\setsansfont{FreeSans}[
  Extension      = .otf,
  UprightFont    = *,
  ItalicFont     = *Oblique,
  BoldFont       = *Bold,
  BoldItalicFont = *BoldOblique,
]
\setmonofont{FreeMono}[
  Extension      = .otf,
  UprightFont    = *,
  ItalicFont     = *Oblique,
  BoldFont       = *Bold,
  BoldItalicFont = *BoldOblique,
]



\usepackage[Bjarne]{fncychap}
\usepackage[,numfigreset=1,mathnumfig]{sphinx}

\fvset{fontsize=\small}
\usepackage{geometry}


% Include hyperref last.
\usepackage{hyperref}
% Fix anchor placement for figures with captions.
\usepackage{hypcap}% it must be loaded after hyperref.
% Set up styles of URL: it should be placed after hyperref.
\urlstyle{same}


\usepackage{sphinxmessages}


\usepackage{etoolbox}
\AtBeginEnvironment{figure}{\pretocmd{\hyperlink}{\protect}{}{}}  

        % Start of preamble defined in sphinx-jupyterbook-latex %
         \usepackage[Latin,Greek]{ucharclasses}
        \usepackage{unicode-math}
        % fixing title of the toc
        \addto\captionsenglish{\renewcommand{\contentsname}{Contents}}
        \hypersetup{
            pdfencoding=auto,
            psdextra
        }
        % End of preamble defined in sphinx-jupyterbook-latex %
        

\title{Coupled Wave Theory of Distributed Feedback Lasers}
\date{May 05, 2023}
\release{}
\author{Jan Genoe, Iakov Goldberg, Nirav Annavarapu}
\newcommand{\sphinxlogo}{\sphinxincludegraphics{logo.png}\par}
\renewcommand{\releasename}{}
\makeindex
\begin{document}

\pagestyle{empty}
\sphinxmaketitle
\pagestyle{plain}
\sphinxtableofcontents
\pagestyle{normal}
\phantomsection\label{\detokenize{intro::doc}}


\sphinxAtStartPar
This jupyter book applies the calculations of Kogelnik and Shank in J. Appl. Phys. 43, 2327 (1972) {[}\hyperlink{cite.bib:id3}{1}{]} to a few use cases.
The calculations have been validated by replotting the graphs from {[}\hyperlink{cite.bib:id3}{1}{]}.
Each of the chapters elaborates one use case.

\sphinxAtStartPar
The online version allows to recalculate all graphs in \sphinxhref{https://jangenoe.github.io/Coupled-Wave-Theory\_DFB-Lasers/Lite}{jupyter\sphinxhyphen{}lite} (even without any code installed).

\sphinxAtStartPar
All code is available on \sphinxhref{https://github.com/jangenoe/Coupled-Wave-Theory\_DFB-Lasers}{Github}

\sphinxstepscope


\chapter{Coupled\sphinxhyphen{}Wave Theory of Distributed Feedback Lasers}
\label{\detokenize{Kogelnik-Shank_Coupled-Wave-Theory_DFB-Lasers:coupled-wave-theory-of-distributed-feedback-lasers}}\label{\detokenize{Kogelnik-Shank_Coupled-Wave-Theory_DFB-Lasers::doc}}
\sphinxAtStartPar
This page comprises simulations from the manuscript H. Kogelnik and C. V. Shank, \sphinxstyleemphasis{Coupled‐Wave Theory of Distributed Feedback Lasers}. Journal of Applied Physics, \sphinxstylestrong{43} (1972) 2327–2335. \sphinxhref{https://doi.org/10.1063/1.1661499}{doi:10.1063/1.1661499}

\sphinxAtStartPar
We compare the different simulations in the paper with our code.

\begin{sphinxuseclass}{cell}
\end{sphinxuseclass}
\begin{sphinxuseclass}{cell}
\end{sphinxuseclass}

\section{Dispersion diagram for index modulation for various gain to coupling parameter ratios}
\label{\detokenize{Kogelnik-Shank_Coupled-Wave-Theory_DFB-Lasers:dispersion-diagram-for-index-modulation-for-various-gain-to-coupling-parameter-ratios}}
\sphinxAtStartPar
\hyperref[\detokenize{Kogelnik-Shank_Coupled-Wave-Theory_DFB-Lasers:kogelnik1}]{Fig.\@ \ref{\detokenize{Kogelnik-Shank_Coupled-Wave-Theory_DFB-Lasers:kogelnik1}}} (left) calculates the dispersion diagram for index modulation for various gain (\(\alpha_o\)) to coupling (\(\kappa\)) parameter ratios. In case of index modulation we have that \(\kappa= \pi n_1/\lambda_o\). We observe that the calculated result fits the result from {[}\hyperlink{cite.bib:id3}{1}{]}, as shown in \hyperref[\detokenize{Kogelnik-Shank_Coupled-Wave-Theory_DFB-Lasers:kogelnik1}]{Fig.\@ \ref{\detokenize{Kogelnik-Shank_Coupled-Wave-Theory_DFB-Lasers:kogelnik1}}} (right).

\begin{sphinxuseclass}{cell}\begin{sphinxVerbatimOutput}

\begin{sphinxuseclass}{cell_output}
\begin{figure}[htbp]
\centering
\capstart

\noindent\sphinxincludegraphics{{19a6cc7e1fd073f0d4c165d56e37007f1f055eb8bce23f54943d10f06c612e70}.png}
\caption{Left: calculated dispersion diagram for index modulation for various gain to coupling parameter ratios, right: corresponding graph from from {[}\hyperlink{cite.bib:id3}{1}{]}}\label{\detokenize{Kogelnik-Shank_Coupled-Wave-Theory_DFB-Lasers:kogelnik1}}\end{figure}

\end{sphinxuseclass}\end{sphinxVerbatimOutput}

\end{sphinxuseclass}

\section{Dispersion diagram for gain modulation for various gain to coupling parameter ratios}
\label{\detokenize{Kogelnik-Shank_Coupled-Wave-Theory_DFB-Lasers:dispersion-diagram-for-gain-modulation-for-various-gain-to-coupling-parameter-ratios}}
\sphinxAtStartPar
\hyperref[\detokenize{Kogelnik-Shank_Coupled-Wave-Theory_DFB-Lasers:kogelnik3}]{Fig.\@ \ref{\detokenize{Kogelnik-Shank_Coupled-Wave-Theory_DFB-Lasers:kogelnik3}}} (left) calculates the dispersion diagram for index modulation for various gain (\(\alpha_o\)) to coupling (\(\kappa\)) parameter ratios. In case of index modulation we have that \(\kappa= \frac{1}{2} j \alpha_1\). We observe that the calculated result fits the result from {[}\hyperlink{cite.bib:id3}{1}{]}, as shown in \hyperref[\detokenize{Kogelnik-Shank_Coupled-Wave-Theory_DFB-Lasers:kogelnik3}]{Fig.\@ \ref{\detokenize{Kogelnik-Shank_Coupled-Wave-Theory_DFB-Lasers:kogelnik3}}} (right).

\begin{sphinxuseclass}{cell}\begin{sphinxVerbatimOutput}

\begin{sphinxuseclass}{cell_output}
\begin{figure}[htbp]
\centering
\capstart

\noindent\sphinxincludegraphics{{f9d2ec31ba1962c1563a8457be1428c31c2da641335ca7fb0faf8ff6e210a722}.png}
\caption{Left: calculated dispersion diagram for gain modulation for various gain to coupling parameter ratios, right: corresponding graph from from {[}\hyperlink{cite.bib:id3}{1}{]}}\label{\detokenize{Kogelnik-Shank_Coupled-Wave-Theory_DFB-Lasers:kogelnik3}}\end{figure}

\end{sphinxuseclass}\end{sphinxVerbatimOutput}

\end{sphinxuseclass}

\section{Mode spectrum for index coupling}
\label{\detokenize{Kogelnik-Shank_Coupled-Wave-Theory_DFB-Lasers:mode-spectrum-for-index-coupling}}
\begin{sphinxuseclass}{cell}\begin{sphinxVerbatimOutput}

\begin{sphinxuseclass}{cell_output}
\begin{figure}[htbp]
\centering
\capstart

\noindent\sphinxincludegraphics{{4dab20e55683836fb199def5b96feb348d5827e1db6f4c5f53842dcf11e23e97}.png}
\caption{Left: calculated gain required for threshold vs frequency deviation from the Bragg condition for index modulation.  Only half of the spectrum is shown because of symmetry, right: corresponding graph from from {[}\hyperlink{cite.bib:id3}{1}{]}}\label{\detokenize{Kogelnik-Shank_Coupled-Wave-Theory_DFB-Lasers:kogelnik5c}}\end{figure}

\end{sphinxuseclass}\end{sphinxVerbatimOutput}

\end{sphinxuseclass}

\section{Mode spectrum for gain coupling}
\label{\detokenize{Kogelnik-Shank_Coupled-Wave-Theory_DFB-Lasers:mode-spectrum-for-gain-coupling}}
\begin{sphinxuseclass}{cell}\begin{sphinxVerbatimOutput}

\begin{sphinxuseclass}{cell_output}
\begin{figure}[htbp]
\centering
\capstart

\noindent\sphinxincludegraphics{{853019762eace1f33512640bcf37157825fddb14a0d0f3549cf1d7c5fa44f1ee}.png}
\caption{Left: calculated DC gain required for threshold vs frequency deviation from the Bragg condition for gain modulation. Only half of the spectrum is shown because of symmetry, right: corresponding graph from from {[}\hyperlink{cite.bib:id3}{1}{]}}\label{\detokenize{Kogelnik-Shank_Coupled-Wave-Theory_DFB-Lasers:kogelnik7c}}\end{figure}

\end{sphinxuseclass}\end{sphinxVerbatimOutput}

\end{sphinxuseclass}
\begin{sphinxuseclass}{cell}\begin{sphinxVerbatimOutput}

\begin{sphinxuseclass}{cell_output}
\begin{figure}[htbp]
\centering
\capstart

\noindent\sphinxincludegraphics{{0f541027450e7ebfdf9ecd91a73353ae824a31803bb356bf9b293237394a7f35}.png}
\caption{Left: calculated gain at threshold vs coupling strength for various modes, right: corresponding graph from from {[}\hyperlink{cite.bib:id3}{1}{]}}\label{\detokenize{Kogelnik-Shank_Coupled-Wave-Theory_DFB-Lasers:kogelnik8c}}\end{figure}

\end{sphinxuseclass}\end{sphinxVerbatimOutput}

\end{sphinxuseclass}
\begin{sphinxuseclass}{cell}\begin{sphinxVerbatimOutput}

\begin{sphinxuseclass}{cell_output}
\begin{figure}[htbp]
\centering
\capstart

\noindent\sphinxincludegraphics{{3542270cdb49302b110bf945ef3d678d54d3e0d3e0841236a5fddb54f78c7531}.png}
\caption{Left: calculated gain at threshold vs coupling strength for various modes. The mode number N refers to a set of modes placed symmetrically about the Bragg frequency, right: corresponding graph from from {[}\hyperlink{cite.bib:id3}{1}{]}}\label{\detokenize{Kogelnik-Shank_Coupled-Wave-Theory_DFB-Lasers:kogelnik9c}}\end{figure}

\end{sphinxuseclass}\end{sphinxVerbatimOutput}

\end{sphinxuseclass}

\section{Spatial intensity distribution of the different modes}
\label{\detokenize{Kogelnik-Shank_Coupled-Wave-Theory_DFB-Lasers:spatial-intensity-distribution-of-the-different-modes}}
\sphinxAtStartPar
In this section we plot a few of the spatial intensity distribution of the different lowest order modes. These intensity distributions are relevant when continuous operation is targeted, as they indicate how the electrical or optical pumping needs to be distributed to maintain lasing.

\begin{sphinxuseclass}{cell}\begin{sphinxVerbatimOutput}

\begin{sphinxuseclass}{cell_output}
\begin{figure}[htbp]
\centering
\capstart

\noindent\sphinxincludegraphics{{d1b598b4786be9b8313aeadd254d6c034b76e77423ef3c734b92d291c15e9d72}.png}
\caption{Left: calculated  spatial intensity distribution of the lowest order modes at various coupling levels, right: corresponding graph from from {[}\hyperlink{cite.bib:id3}{1}{]}}\label{\detokenize{Kogelnik-Shank_Coupled-Wave-Theory_DFB-Lasers:kogelnik10c}}\end{figure}

\end{sphinxuseclass}\end{sphinxVerbatimOutput}

\end{sphinxuseclass}
\begin{sphinxuseclass}{cell}\begin{sphinxVerbatimOutput}

\begin{sphinxuseclass}{cell_output}
\begin{figure}[htbp]
\centering
\capstart

\noindent\sphinxincludegraphics{{34f9bfc9b0dd56d4135ba99bb1a1f44ed493b321fb6d2d6511a411028d5678e1}.png}
\caption{Left: calculated spatial intensity distribution for the first three modes, right: corresponding graph from from {[}\hyperlink{cite.bib:id3}{1}{]}}\label{\detokenize{Kogelnik-Shank_Coupled-Wave-Theory_DFB-Lasers:kogelnik11ac}}\end{figure}

\end{sphinxuseclass}\end{sphinxVerbatimOutput}

\end{sphinxuseclass}
\begin{sphinxuseclass}{cell}\begin{sphinxVerbatimOutput}

\begin{sphinxuseclass}{cell_output}
\begin{figure}[htbp]
\centering
\capstart

\noindent\sphinxincludegraphics{{3ccdee0f554eb1cf29730b5289552e5a7ff495d94667f8c70893f90e149e2909}.png}
\caption{Left: calculated spatial intensity distribution for the first three modes, right: corresponding graph from from {[}\hyperlink{cite.bib:id3}{1}{]}}\label{\detokenize{Kogelnik-Shank_Coupled-Wave-Theory_DFB-Lasers:kogelnik11bc}}\end{figure}

\end{sphinxuseclass}\end{sphinxVerbatimOutput}

\end{sphinxuseclass}
\begin{sphinxuseclass}{cell}\begin{sphinxVerbatimOutput}

\begin{sphinxuseclass}{cell_output}
\begin{figure}[htbp]
\centering
\capstart

\noindent\sphinxincludegraphics{{70e5a6e523d9b76cdf8a73eb77f16e60572b14275ccb39e9639874a5ceff9c45}.png}
\caption{Left: calculated spatial intensity distribution for the first three modes, right: corresponding graph from from {[}\hyperlink{cite.bib:id3}{1}{]}}\label{\detokenize{Kogelnik-Shank_Coupled-Wave-Theory_DFB-Lasers:kogelnik11cc}}\end{figure}

\end{sphinxuseclass}\end{sphinxVerbatimOutput}

\end{sphinxuseclass}
\begin{sphinxuseclass}{cell}\begin{sphinxVerbatimOutput}

\begin{sphinxuseclass}{cell_output}
\begin{figure}[htbp]
\centering
\capstart

\noindent\sphinxincludegraphics{{b0dd10c53bab37a6e60b2489534dc65827fbd016cdb2f08a8b3a4bddcc593d87}.png}
\caption{Left: calculated spatial intensity distribution for the first three modes, right: corresponding graph from from {[}\hyperlink{cite.bib:id3}{1}{]}}\label{\detokenize{Kogelnik-Shank_Coupled-Wave-Theory_DFB-Lasers:kogelnik11dc}}\end{figure}

\end{sphinxuseclass}\end{sphinxVerbatimOutput}

\end{sphinxuseclass}
\begin{figure}[htbp]
\centering
\capstart

\noindent\sphinxincludegraphics{{Kogelnik12a}.png}
\caption{Plot of the spatial intensity distribution for the first three modes at \(\alpha L / 2 =0.25\). From {[}\hyperlink{cite.bib:id3}{1}{]}}\label{\detokenize{Kogelnik-Shank_Coupled-Wave-Theory_DFB-Lasers:kogelnik12a}}\end{figure}

\begin{sphinxuseclass}{cell}\begin{sphinxVerbatimOutput}

\begin{sphinxuseclass}{cell_output}
\begin{figure}[htbp]
\centering
\capstart

\noindent\sphinxincludegraphics{{ce2c1c36764283776e8d6d3d5dd305fb94206688b2a35b694830e18e0517edab}.png}
\caption{Left: calculated spatial intensity distribution for the first three modes, right: corresponding graph from from {[}\hyperlink{cite.bib:id3}{1}{]}}\label{\detokenize{Kogelnik-Shank_Coupled-Wave-Theory_DFB-Lasers:kogelnik12bc}}\end{figure}

\end{sphinxuseclass}\end{sphinxVerbatimOutput}

\end{sphinxuseclass}
\begin{sphinxuseclass}{cell}\begin{sphinxVerbatimOutput}

\begin{sphinxuseclass}{cell_output}
\begin{figure}[htbp]
\centering
\capstart

\noindent\sphinxincludegraphics{{07134a6f25fc53c491095eabfb342f9682c44dcbcea1c64982b857479d032983}.png}
\caption{Left: calculated spatial intensity distribution for the first three modes, right: corresponding graph from from {[}\hyperlink{cite.bib:id3}{1}{]}}\label{\detokenize{Kogelnik-Shank_Coupled-Wave-Theory_DFB-Lasers:kogelnik12cc}}\end{figure}

\end{sphinxuseclass}\end{sphinxVerbatimOutput}

\end{sphinxuseclass}
\begin{sphinxuseclass}{cell}\begin{sphinxVerbatimOutput}

\begin{sphinxuseclass}{cell_output}
\begin{figure}[htbp]
\centering
\capstart

\noindent\sphinxincludegraphics{{a4f0ff42affcba59bc7d8cf0261faab986725da7e9838a1f557c71ca9a4c2e98}.png}
\caption{Left: calculated spatial intensity distribution for the first three modes, right: corresponding graph from from {[}\hyperlink{cite.bib:id3}{1}{]}}\label{\detokenize{Kogelnik-Shank_Coupled-Wave-Theory_DFB-Lasers:kogelnik12dc}}\end{figure}

\end{sphinxuseclass}\end{sphinxVerbatimOutput}

\end{sphinxuseclass}
\sphinxstepscope


\chapter{UV\sphinxhyphen{}Nanoimprinted Distributed\sphinxhyphen{}Feedback Perovskite Lasing}
\label{\detokenize{NanoimprintedDFB:uv-nanoimprinted-distributed-feedback-perovskite-lasing}}\label{\detokenize{NanoimprintedDFB::doc}}
\sphinxAtStartPar
This chapter comprises simulations from the manuscript:

\sphinxAtStartPar
Iakov Goldberg, Nirav Annavarapu, Simon Leitner, Karim Elkhouly, Fei Han, Tibor Kuna, Weiming Qiu, Cedric Rolin, Jan Genoe, Robert Gehlhaar and Paul Heremans, \sphinxstyleemphasis{“Multimode Lasing in All\sphinxhyphen{}Solution\sphinxhyphen{}Processed UV\sphinxhyphen{}Nanoimprinted Distributed Feedback MAPbI$_{\text{3}}$ Perovskite Waveguides”}, submitted manuscript

\sphinxAtStartPar
The calculation of the modes is done in accordance to {[}\hyperlink{cite.bib:id3}{1}{]}.

\begin{sphinxuseclass}{cell}
\end{sphinxuseclass}
\begin{sphinxuseclass}{cell}\begin{sphinxVerbatimOutput}

\begin{sphinxuseclass}{cell_output}
\begin{figure}[htbp]
\centering
\capstart

\noindent\sphinxincludegraphics{{1ebcf8af2a8ef5af869371d1fe3ed947532972361e69aa275d2eeef3280d24c8}.png}
\caption{Upper left: calculated dispersion relation prior to lasing, Upper right: intensity of the lasing mode in the near field, lower left: required DC gain for the lasing threshold, the dotted lines show the measured lasing lines at 3 different locations, lower right: far field lasing intensity as a function of the angle}\label{\detokenize{NanoimprintedDFB:iakov1}}\end{figure}

\end{sphinxuseclass}\end{sphinxVerbatimOutput}

\end{sphinxuseclass}
\sphinxstepscope

\begin{sphinxthebibliography}{1}
\bibitem[1]{bib:id3}
\sphinxAtStartPar
H. Kogelnik and C. V. Shank. Coupled\sphinxhyphen{}Wave Theory of Distributed Feedback Lasers. \sphinxstyleemphasis{Journal of Applied Physics}, 43(5):2327–2335, May 1972. \sphinxhref{https://doi.org/10.1063/1.1661499}{doi:10.1063/1.1661499}.
\end{sphinxthebibliography}







\renewcommand{\indexname}{Index}
\printindex
\end{document}